%Oficina de \LaTeX: ``Meu primeiro artigo''. Hands-on.

\section{Orientações gerais}

%%
\begin{frame}{Nota}
As próximas 3 aulas (08, 09 e 10, no plano de curso) são dedicadas a uma oficina prática com foco em produção de artigos científicos de alto desempenho com {\LaTeX} através da plataforma \texttt{Overleaf}. 

O roteiro do mini-curso, bem como o material utilizado está disponível no repositório: 

\url{https://github.com/gcpeixoto/mc-latex}

\end{frame}

%%
\begin{frame}{Antes de começar...} 

Vide arquivo \texttt{motivacao.pdf} no diretório \texttt{motivacao} do repositório acima. 

Vide arquivo \texttt{guide.pdf} no diretório \texttt{guide} do repositório acima. 

\end{frame}

\section{Conteúdo do mini-curso} 

\begin{frame}

Introdução à linguagem tipográfica {\LaTeX}. Ambientes de compilação, edição e compartilhamento online. Edição básica de conteúdo (elementos textuais diversos, fórmulas matemáticas, referências cruzadas, tabelas e figuras). Ênfase em preparação de artigos científicos. Noções de Markdown e Mathjax.

\end{frame}